\section{Функции распределения на $\mathbb{R}$}

\subsection{Вероятностное пространство}
Тройка
$
(\Omega, \mathcal{F}, P)
$
называется вероятностным пространством.

\begin{enumerate}
	\item $ \Omega $ - пространство элементарных событий
	\item $ \mathcal{F} $ - $ \sigma\text{-алгебра}  $  подмножеств $ \Omega $
\end{enumerate}

\begin{definition}
Произвольны элемент	$ A \in \mathcal{F} $ называвается \textit{событием}
\end{definition}

\begin{definition}
	Система множеств $\mathcal{A}$ называется
	алгеброй, если
	\begin{enumerate}
		\item $\Omega \in \mathcal{A}$
		\item если $ A \in \mathcal{A}   \Longrightarrow  
		\overline{A} \in \mathcal{A} $
		\item $ A,B \in \mathcal{A} \Longrightarrow A \cup B \in \mathcal{A}
		 $
	\end{enumerate}	
	
\end{definition}	


\begin{definition}
	Система подмножеств $ \mathcal{F}  $ называется $ \sigma\text{-алгеброй}  $ если
	\begin{enumerate}
	\item $\Omega \in \mathcal{F}$
	\item если $ A \in \mathcal{F}   \Longrightarrow  
	\overline{A} \in \mathcal{F} $
	\item
	$\displaystyle \lbrace A_i \rbrace^{\infty}_{i=0} \in \mathcal{F} \Longrightarrow \bigcup^{\infty}_{i=0} A_i \in \mathcal{F}$
	\end{enumerate}
\end{definition}

\begin{definition}
	Пара $ (\mathcal{F}, \Omega) $ - измеримое пространство
\end{definition}	


\begin{definition}
	Отображение $ P:\mathcal{F} \rightarrow \left[0,1\right]$ - называется вероятностной мерой на $ (\mathcal{F}, \Omega) $ если
	\begin{enumerate}
	\item $ P(\Omega)=1 $
	\item $\displaystyle \forall  \lbrace A_i \rbrace^{\infty}_{i=0}  :
	\forall i \neq j A_{i} \cap A_{j}=\varnothing \hookrightarrow P\left(\bigsqcup_{i=1}^{\infty} A_{i}\right)=\sum_{i=1}^{\infty} P\left(A_{i}\right)
	$
	\end{enumerate}	
\end{definition}

\begin{properties}
	\begin{enumerate}
		\item $P(\varnothing)=0$
		\item $A\cap B=\varnothing \rightarrow P(AB) = P(A) \cdot P(B)$
		\item $P(\overline{A})=1-P(A)$
		\item $P(A \cup B)=P(A)+P(B)-P(A \cap B)$
		\item $A \subseteq B \rightarrow  \mathrm{P}(A) \leq \mathrm{P}(B)$
	\end{enumerate}
\end{properties}

\begin{definition}
	Последовательность событий $ \lbrace A_n \rbrace_{n \geq 1 } $ убывает к $ A $, $ (A_n\downarrow A) $, если 
	\begin{enumerate}
		\item $  \forall n \in \mathbb{N} \quad A_{n} \supseteq A_{n+1}$
		\item $\displaystyle A=\bigcap_{n=1}^{\infty} A_{n}$
	\end{enumerate}	
\end{definition}	

\begin{definition}
		Последовательность событий $ \lbrace A_n \rbrace_{n \geq 1 } $ возрастает к $ A $, $ (A_n\uparrow A) $, если 
	\begin{enumerate}
		\item $  \forall n \in \mathbb{N} \quad A_{n} \subseteq A_{n+1}$
		\item $\displaystyle A=\bigcup_{n=1}^{\infty} A_{n}$
	\end{enumerate}	
\end{definition}

\begin{theorem}[о непрерывности вероятностной меры]
	Вероятность $ (\Omega, \mathcal{F})$ -измеримое пространство а $ P $ удоволетворяет свойствам 
	\begin{enumerate}
		\item $ P(\Omega) = 1 $
		\item $ P - \text{конечно-аддитивна} $
	\end{enumerate}	

Тогда $ P $ - вероятностная мера $\Longleftrightarrow$ $ P $ непрерывна в $ 0 $ $ \left((A_{n} \downarrow \varnothing),\hspace{3pt} \text{то} \hspace{3pt} P\left(A_{n}\right) \rightarrow 0\right) $
\end{theorem}

\subsection{Верояностноя мера на $\mathbb{R}$}
Пусть $ P $  вероятностноя мера на $(\mathbb{R}, \mathfrak{B}(\mathbb{R}))$
\begin{definition}
Функция $ F(x), \quad x\in \mathbb{R}  $ определённая по правилу
$F(x)=P\left(\left(-\infty, x\right]\right)$ называется функицией распределения вероятностной меры $ P $
\end{definition}
\begin{lemma}[Cвойства функции распределения]
	Пусть $F(x)$ - функция распределения на $ \mathbb{R} $ вероятносной меры $ P $
	Тогда 
	\begin{enumerate}
		\item	$ F(x) $ -не убывает
		\item $\displaystyle \lim _{x \rightarrow-\infty} F(x)=0 \lim _{x \rightarrow+\infty} F(x)=1$
		\item $ F(x) $ непрерывна справа		
	\end{enumerate}
	
\end{lemma}

\begin{proof}
	\begin{enumerate}
		\item Пусть $ y \geq x $. Тогда $F(y)-F(x) = P((-\infty, y])-P(-\infty, x]) =  P(\left[x,y\right]) \geq 0$
		
		\item 
		Пусть $x_{n} \rightarrow-\infty$ при $n \rightarrow \infty$. Тогда $\left(\left(-\infty, x_{0}\right] \downarrow \varnothing\right)  \Rightarrow тогда по непрерывности верояностной меры F(x_n) = P\left(-\infty,x_n\right) \rightarrow P(\varnothing)=0 $
		
		
		Аналогично, если $\left(-\infty, x_{n}\right] \uparrow \mathbb{R}$, то $F\left(x_{n}\right) \rightarrow 1$
		
		\item 
		Пусть $x_{n}\downarrow x+0$, $\left(-\infty, x_{n}\right] \rightarrow(-\infty, x)$ $\Rightarrow$ по непрерывности вероятностной меры,
		$F\left(x_{n}\right)=P((-\infty, x_n ) \rightarrow P((-\infty, x])=F(x)$
		
	\end{enumerate}	
\end{proof}	

\begin{definition}
	Любая функция $ F(x) $ удоволетворяющая свойства \textbf{Леммы 1} является функцией распределения
\end{definition}

\begin{definition}[Кольцо множеств]
	Непустая система множеств $\mathfrak{R}$ называется \textit{кольцом}, если она обладает тем свойством, что из $ A \in \mathfrak{R} $ и $ B \in  \mathfrak{R}$ следует $A \Delta B \in \mathfrak{R}$ и $A \cap B \in \mathfrak{R}$
\end{definition}
\begin{definition}[Полукольцо множеств]
	Система множеств $\mathfrak{S}$ называется \textit{кольцом}, если она cодержит пустное множестов $ \varnothing $, замкнута по отно- 
	шению к образованию пересечений и обладает тем свойством, что 
	из принадлежности к $\mathfrak{S}$ множеств$A \textbf{и} A_{1} \subset A$ вытекает возможность представления $A$ в виде $A=\bigcup_{k=1}^{n} A_{k}$, где $A_{k}$ попарно непересекающиеся множества из $\mathfrak{S}$, первое из которых есть заданное 
	множество $A_{1}$.
\end{definition}		
 
\begin{theorem}[Теорема Каратеодори]
	Пусть $ \Omega $ некоторое множество, $ S $ -полукольцо на $ \Omega $. $ P_{\sigma} $ вероятностная мера на $ \left(\Omega, S\right) $. Тогда $ \exists ! $ верояностная мера $ P $
	на $ (\Omega ,\sigma(S) )$, является продолжение меры $ P_{\sigma} $  $ \left(\forall A \in S, P_{\sigma} = P \right) $
\end{theorem}


\begin{theorem}[Теорма о взаимооднозначном соответсвтии функций распределения и вероятностных мер]	
Пусть $ F(x) $ - функция распределения на $ \mathbb{R} $ Тогда $ \exists ! $ вероятностная мера на $ (\mathbb{R},\mathfrak{B}(\mathbb{R})): F(x) является ей функция распределения т.е. F(x)=P(\left(-\infty, x\right]) $	
\end{theorem}

\subsection{Классификация вероятностных моделей и функций распределения на $ (\mathbb{R},\mathfrak{B}(\mathbb{R}))$}

\begin{enumerate}
	\item{Дискретное распределение}
	Пусть на $ X \subseteq \mathbb{R}$ не более чем счётное множество
	\begin{definition}
		Вероятностная мера $ P $ на $ (\mathbb{R},\mathfrak{B}(\mathbb{R}))$ удоволетворяет свойству $P(\mathbb{R} / X)=0$ называется дискретной мерой на $ X $
	\end{definition}	
	\item	
\end{enumerate}

